\documentclass{article}
\usepackage{amsmath}
\usepackage{graphicx}
\usepackage[spanish,es-tabla]{babel}
\usepackage[utf8]{inputenc}
 
\begin{document}
\begin{titlepage}
\centering
{\bfseries\LARGE Universidad del Cauca \par}
\vspace{1cm}
{\scshape\Large Facultad de Ingeniería Civil\par}
\vspace{1cm}
{\scshape\Huge Ecuaciones Diferenciales\par}
{\bfseries\LARGE  “ Naturaleza de las soluciones, problemas de inicio, de contorno y de existencia de soluciones ” \par}
\begin{figure}
    \centering
    \includegraphics[width=0.4\textwidth]{logo}
    \label{fig:my_label}
\end{figure}
\vspace{3cm}
{\itshape\Large presentado a: Jhonatan Collazos \par}
\vfill
{\itshape\Large presentado por:  Daniela Fernanda Reuto  \par}
\vfill
{\Large 22 de Agosto del 2022 \par}
\end{titlepage}
\tableofcontents
\newpage
\section{Introducción}
Una ecuación diferencial es aquella que relaciona las variables independientes con la variable dependiente y sus derivadas con respecto a una o más variables independientes. Las ecuaciones diferenciales juegan un papel fundamental tanto en la propia Matemática como en otras ciencias como la Física, Química, Economía, Biología, etc \\
Las soluciones de una ecuación diferencial, serán funciones. Por lo tanto, resolver una ecuación diferencial es encontrar una función que junto con sus derivadas satisfacen la ecuación dada.\\
Por último decimos que hay métodos numéricos para resolver ecuaciones diferenciales. Aquí resolver significa dar una aproximación a la solución de la ecuación diferencial.
\section{Marco Teórico}
\subsection{Ecuaciones Diferenciales}
Una ecuación que contiene derivadas de una o más variables respecto a una o
más variables independientes, se dice que es una ecuación diferencial (ED).
Cuando la función desconocida depende de dos o más variables, entonces las derivadas que aparecen en la ecuación diferencial serán derivadas parciales, y en este caso diremos que se trata de una ecuación en derivadas parciales. Si la función depende solo de una variable independiente, entonces la ecuación recibe el nombre de ecuación diferencial ordinaria (E.D.O.).
\subsection{Naturaleza de las soluciones de las ecuaciones diferenciales }
Una solución de una ecuación diferencial es una función que al reemplazar a la función incógnita, en cada caso con las derivaciones correspondientes, verifica la ecuación, es decir, la convierte en una identidad. 
\paragraph{Solución general}
Solución de la ecuación diferencial en la que aparecen tantas constantes arbitrarias como indica el orden de la ecuación. En nuestro caso, al ser de primer orden, la solución general sería una constante arbitraria.
\paragraph{Solución particular}
Es una solución que se obtiene al fijar los valores de las constantes arbitraria de la solución general, en nuestro caso, al fijar el valor de la constante arbitraria C. 
\paragraph{Solución singular}
Es una solución que no está incluida en la solución general; es decir, no se puede obtener a partir de ella asignando un valor conveniente a la constante.\\ 
La solución de una ecuación diferencial puede venir dada de tres formas distintas\\ 
1. En forma explícita si la incógnita viene despejada en función de la variable independiente x.\\
2. En forma implícita si la solución viene expresada por una ecuación que liga la incógnita y y la variable independiente x.\\
3. En forma paramétrica si la solución viene dada en función de un parámetro. 

\paragraph{Ejemplo de una solución general}\cite{Baz}
\begin{equation}
y'= e^{3x}-x
\end{equation}
\begin{equation}
\frac{dy}{dx}=e^{3x}-x
\end{equation}
\begin{equation}
{dy}=(e^{3x}-x){dx}
\end{equation}
\begin{equation}
\int{dy}=\int(e^{3x}){dx}-\int(x{dx})
\end{equation}
\begin{equation}
y=\frac{1}{3}e^{3x}-\frac{x^{2}}{2} + C
\end{equation}
\subsection{Problema de valor inicial}
Un problema que busca determinar una solución particular a una ecuación diferencial sujeta a condiciones sobre la función desconocida y sus derivadas especificadas en un único valor de la variable independiente se denomina problema de valor inicial. Tales condiciones se llaman condiciones iniciales.
\subsection{Problema de valor de contorno}
Un problema que busca determinar una solución particular a una ecuación diferencial sujeta a condiciones sobre la función desconocida y sus derivadas especificadas en dos o más valores (distintos) de la variable independiente se denomina problema de valor frontera o contorno (o problema de contorno). Tales condiciones se llaman condiciones de frontera o (condiciones de contorno)
\paragraph{Nota}
La diferencia entre un problema de valor inicial y un problema de valor de frontera, es que en el problema de valor inicial la función desconocida y sus derivadas se valoran en un mismo valor de la variable independiente, en cambio un problema de frontera se evalúa en al menos en dos valores distintos de la variable independiente.\\
Para resolver un problema de valor inicial o un problema de valor de frontera primero se halla la solución general de la ecuación que describe el problema y luego se sustituye las condiciones según corresponda, para determinar el valor de los parámetros y así obtener la solución particular
\subsection{Teorema de existencia y unicidad}
Establece las condiciones necesarias y suficientes para que una ecuación diferencial de primer orden, con condición inicial dada, tenga una solución y que además dicha solución sea la única.
Sin embargo el teorema no da ninguna técnica ni indicación de cómo hallar tal solución. El teorema de existencia y unicidad se extiende también a ecuaciones diferenciales de orden superior con condiciones iniciales, lo que se conoce como problema de Cauchy.
\paragraph{Ejemplo de solución particular y  problema de inicio}\cite{Dan}
 Sea la ecuación diferencial \emph{$y'' - y'=20$} cuya solución general es \emph{$y=C_1 + C_2e^{t} - 20t$} obtenga la solución particular sujeta a las condiciones iniciales $y(0)=0$  y $y'(0)=40$ 
 \paragraph{Derivamos la solución general}
\begin{equation}
y=C_1 + C_2e^{t} - 20t 
\end{equation}
\begin{equation}
y'=C_2e^{t} - 20t
\end{equation}
\begin{equation}
y''=C_2e^{t}
\end{equation}
\paragraph{Comprobamos que si son soluciones de la ecuación diferencial}
\begin{equation}
y'' - y'=20 
\end{equation}
\begin{equation}
c_2e^{t} - (C_2e^{t} - 20)=20 
\end{equation}
\begin{equation}
20=20 
\end{equation}
\paragraph{Calcular $C_1$ Y $C_2$}
\begin{equation}
y(0)=0 \ \ \textup{donde t=0 y y=0}
\end{equation}
\paragraph{Reemplazamos en la solución general}
\begin{equation}
y=C_1 + C_2e^{t} - 20t 
\end{equation}
\begin{equation}
0=C_1 + C_2e^{0} - 20(0) 
\end{equation}
\begin{equation}
0=C_1 + C_2
\end{equation}
\begin{equation}
C_1=-C_2 
\end{equation}\ \ \
\begin{equation}
y'(0)=40 \ \ \textup{donde t=0 y y'=40}
\end{equation}
\paragraph{Reemplazamos en la primera derivada}
\begin{equation}
y'=C_2e^{t} - 20t
\end{equation}
\begin{equation}
40=C_2e^{0} - 20(0)
\end{equation}
\begin{equation}
40'+ 20 =C_2
\end{equation}
\begin{equation}
60=C_2 \ \ \textup{y}\ \ \ C_1=-60
\end{equation}
\paragraph{Encontramos la solución particular}
\begin{equation}
y=-60+ 60e^{t} - 20t 
\end{equation}
\section{Ejercicio de Aplicación}
Este ejercicio parte de la ley de enfriamiento de Newton donde  
\begin{equation}
T=Tm + e^{kt}C\ \ \textup{Función de temperatura o ley de enfriamiento}
\end{equation}
Esta ley nos dice que la rapidez con la que cambia la temperatura de un objeto es proporcional a la diferencia de la temperatura del objeto y la del medio que lo rodea
y está se representa por medio de una ecuación diferencial 
\begin{equation}
\frac{dT}{dt}=k(T-Tm)
\end{equation}
donde k= constante,
T= temperatura, 
Tm= temperatura del medio, 
t= tiempo.\ \ \

La función (23) sale de despejar T en la ecuación diferencial (24)
\begin{equation}
\frac{dT}{dt}=k(T-Tm)
\end{equation}
\begin{equation}
{dT}=k(T-Tm){dt}
\end{equation}
\begin{equation}
\frac{dT}{T-Tm}=k{dt}
\end{equation}
\begin{equation}
\int\frac{dT}{T-Tm}=\int{k{dt}}
\end{equation}
\begin{equation}
\ln{(T-Tm)}={kt} + C
\end{equation}
\begin{equation}
e^{\ln{(T-Tm)}}=e^{kt + C}
\end{equation}
\begin{equation}
{T-Tm}=e^{kt}C
\end{equation}
\begin{equation}
T=Tm + e^{kt}C
\end{equation}
Con la información brindada acerca de los aportes de Isaac Newton nuestro ejercicio de aplicación lo realizamos con respecto a la temperatura de la mezcla asfáltica que es aproximadamente de $160^{\circ} C$. Suponemos que somos los ingenieros de una obra y tenemos algunos datos como temperatura de la mezcla asfáltica, temperatura del medio ambiente y la medición de la temperatura de la mezcla en un tiempo determinado después de la aplicación y queremos saber cual es la temperatura de la mezcla en un segundo tiempo determinado después de su aplicación. Los datos son hipotéticos pero veremos si tiene sentido el resultado de acuerdo a lo que establece la ley de Newton\\
$T(0)=160^{\circ} C$\\
$Tm=25^{\circ} C$\\
$T(30)=100^{\circ} C$\\
$T(60)=?$\\
\paragraph{Encontrar la constante C}
\begin{equation}
T=Tm + e^{kt}C
\end{equation}
\begin{equation}
160=25 + e^{k(0)}C
\end{equation}
\begin{equation}
160-25=C
\end{equation}
\begin{equation}
C=135
\end{equation}
\paragraph{Encontrar la constante k}
\begin{equation}
T=Tm + e^{kt}C
\end{equation}
\begin{equation}
100=25 + e^{k(30)}135
\end{equation}
\begin{equation}
100-25= 135e^{k(30)}
\end{equation}
\begin{equation}
\frac{75}{135}= e^{k(30)}
\end{equation}
\begin{equation}
\ln(\frac{75}{135})=\ln ^{e^{k(30)}}
\end{equation}
\begin{equation}
\ln(\frac{75}{135})=30k
\end{equation}
\begin{equation}
k=\frac{\ln(\frac{75}{135})}{30}
\end{equation}
\begin{equation}
k=-0,196
\end{equation}
\paragraph{Reemplazamos en la ecuación de temperatura las constantes}
\begin{equation}
T=Tm + e^{kt}C
\end{equation}
\begin{equation}
T=25 + e^{-0,196(t)}135
\end{equation}
\paragraph{Como queremos saber la temperatura de la mezcla pasados 60 minutos reemplazamos en la función (46)}
\begin{equation}
T=25 + e^{-0,196(t)}135
\end{equation}
\begin{equation}
T=25 + e^{-0,196(60)}135
\end{equation}
\begin{equation}
T=66,6489^{\circ}
\end{equation}\\
Analizado los resultados podemos ver que la temperatura a los 60 minutos de su aplicación tiene sentido de acuerdo a lo que establece la ley de Newton, ya que por definición la temperatura de cualquier objeto tiende a ser la misma del medio en el que se encuentra a medida que pasa el tiempo.

\section{Ejercicio de Aplicación 2}
Bucaramanga tenía 25,000 personas en 1960 y 50,000 personas 10 años mas tarde. Si el número de personas crece directamente proporcional al número de personas en un instante dado ¿Cuántas personas hay en Bucaramanga 10 años después?\ \ 
\ \ P(0)=25,000 y P(10)=50,000
\begin{equation}
\frac{dP}{dt}=kP 
\end{equation}
\begin{equation}
\frac{dP}{P}=k{dt}
\end{equation}
\begin{equation}
\int\frac{dP}{P}=\int{k{dt}}
\end{equation}
\begin{equation}
\ln{(P)}={kt} + C
\end{equation}
\begin{equation}
e^{\ln{(P)}}=e^{kt + C}
\end{equation}
\begin{equation}
P=e^{kt}C
\end{equation}
\paragraph{Reemplazamos P(0)=25,000 donde t=0 y P=25,000} 
\begin{equation}
P=e^{kt}C
\end{equation}
\begin{equation}
25,000=e^{k(0)}C
\end{equation}
\begin{equation}
C=25,000
\end{equation}
\paragraph{Reemplazamos P(10)=50,000 donde t=10 y P=50,000} 
\begin{equation}
P=e^{kt}C
\end{equation}
\begin{equation}
50,000=e^{k(10)}(25,000)
\end{equation}
\begin{equation}
e^{k(10)}=2
\end{equation}
\begin{equation}
10k=\ln{2}
\end{equation}
\begin{equation}
k=\frac{\ln{2}}{10}
\end{equation}
\paragraph{Como queremos saber cuantas personas hay en Bucaramanga 20 años después, reemplazamos en la función } 
\begin{equation}
P=e^{kt}C
\end{equation}
\begin{equation}
P=e^{\frac{\ln{2}}{10}(20)}(25,000)
\end{equation}
\begin{equation}
P=e^{2\ln{2}}(25,000)
\end{equation}
\begin{equation}
P=e^{\ln{4}}(25,000)
\end{equation}
\begin{equation}
P=(4)(25,000)
\end{equation}
\begin{equation}
P=(4)(25,000)
\end{equation}
\begin{equation}
P=100,000
\end{equation}
\begin{thebibliography}{X}
\bibitem{Baz} \textsc{GEROMATH Ejercicios Matemáticos} \textsc{Hallar la solución general de las ecuaciones diferenciales.}
\bibitem{Dan} \textsc{Profesora Lina}  \textsc{Solución particular de una ecuación diferencial},
\textit{Ecuaciones Diferenciales}
\bibitem{Fer} \textsc{Rosado, Eugenia and EDO, ETS Arquitectura}  \textsc{Rosado, Eugenia and EDO, ETS Arquitectura}
\bibitem{car} \textsc{Zill, Dennis G and Cullen, Michael R}  \textsc{Ecuaciones diferenciales},
\textit{McGraw-Hill} 2008
\bibitem{Ter} \textsc{Cappello, Viviana and Herrera, Romina and others}  \textsc{Antrom{\'a}tica},
\textit{Editorial de la Universidad Nacional de La Plata (EDULP)} 2017
\bibitem{Ter} \textsc{OSCAR PARDO}  \textsc{LEY DE ENFRIAMIENTO DE NEWTON | Problema de aplicación en la Ingeniería Civil},
\end{thebibliography}
\end{document}